\section{Reflection}
The proposed architecture satisfies the base requirement of providing components that interact with each other to produce Lego bricks while satisfying the initial requirements. The sections strives to answer the research questions, whereas it is stated to be answered by the architectural research conducted, but it is often not made explicit.
The first occurrence of this is regarding exploring different architectures. The report explores different arrangements of subsystems, not different architectures. The amount of rearranging done in the report should not be considered a fundamental change in the architecture since it maintains the same general structure and interconnections for most subsystems. In hindsight, it could be solved by changing the wording of the research questions to include reconfiguration or refinement of the software architecture instead. The other research question regarding trade-offs is answered implicitly throughout the report. The trade-offs are only sometimes made evident but are included in the conclusion. The trade-offs are only sometimes made evident but are included in the conclusion which could lead to missing cohesion, making it a minor oversight. The reported experiment aimed to evaluate the proposed solution's quality attributes by focusing on one of the required quality attributes. The initial experiment evaluated the system's interoperability, but the results and execution better described the system's performance. Performance is also a vital quality attribute but not one of the required ones. Even though the report does not test one of the required quality attributes, they are considered in the architectural design. Due to lack of time, the experiment was kept the same and could have quickly been revised if the time had not been limited by devising a new experiment focusing on the right quality attributes.