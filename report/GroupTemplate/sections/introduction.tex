\section{Introduction}
Industry 4.0 has been evolving for many years by the reason of the new technologies introduced i.e., Artificial Intelligence (AI) and Internet of Things (IoT) that can improve a company's production line and many other aspects \cite{9667102}. In the future, new technologies and techniques will be introduced and therefore every decision made in the past should be justified. However, there will always be trade-offs that can have an impact on future implementation. Therefore, software architecture is one of the important domains when designing a complex system where the decisions e.g., design patterns and technologies play a crucial role in the future. The decisions that have been taken in the early stage of development are the key to the future development of the system. This is where the requirements of the system and corresponding quality attributes are the key points, and the quality attributes can decide if the chosen technology or pattern is suitable for the requirement. 

 The project would focus on designing an architecture and implementing a complex system within the Industry 4.0 domain, which is a production line that would recycle used Lego bricks. The complex system has requirements that must be part of the final design of the system, where each decision should be justified and consider the possible trade-offs. The requirements are an essential part that would qualify the design of the software architecture. The research questions would investigate and analyse how the role and types of different software architectures can benefit the complex system while addressing the possible trade-offs of each decision made. This would provide insight and give an in-depth understanding of how the architectural decisions and technologies can secure the best possible architectural design for the specific requirements.
