\begin{abstract}
Architectural design challenges for production software within the industry 4.0 is inevitable and should be considered thoroughly to accommodate a flexible and complex production system. The objective of this study is to design a complex system and to use suitable software architectures as well as technologies through prototype implementation. The research seeks to address how different architectures can support production system requirements, and which architectural tradeoffs must be examined due to technology choices. To assist the qualities of interoperability, deployability and availability, the microservice architecture is used to avoid strict dependencies, while promoting loose coupling and high cohesion. The inclusion of using event-driven architecture allowed the services to respond with events to support efficient communication, while the usage of the publish-subscribe pattern was beneficial to accommodate the setup required for message-oriented middleware. An experiment devised highlighted the hypothesis that message latency between two independent services is lower if the communication is inferred between one broker instead of two, and it should at maximum have one second of lead time. Performance was validated  through a simulated toy model while the key metric to be accounted for was the timeliness of data exchange for end-to-end latency between services. It was confirmed that message latency between two brokers is higher in contrast to only using one. The increase of timeliness were trivial, which stresses the current architectural design being deployed to both be optimal and act in accordance with the stated requirements.
\end{abstract}