\section{Conclusion}
This paper has investigated the architectural design challenges within a flexible and complex production system by conducting an experiment. The objective were to design a complex system within the industry 4.0 domain and investigating suitable software architectures as well as technologies to implement a prototype. The approach were to satisfy production software to exchange and coordinate information to both execute a production and change production respectively, while being continuous deployable and running nonstop. Furthermore, the study investigates how a production system can be designed to assist production components to interact effectively and to collect, sort and package recycled Lego bricks accordingly. To support this basis, the research would focus upon how different architectures can support production system requirements as well as to reflect on which architectural tradeoffs must be examined due to technology choices. To assist the qualities of interoperability, deployability and availability, the prototype utilized a microservice architecture to avoid strict dependencies. It would promote loose coupling and high cohesion whereas the system would be composed of independent services. Furthermore, the adaptation of using event-driven architecture allowed the services to respond with events through a message bus to support efficient communication. In addition, the usage of the publish-subscribe pattern was beneficial to accommodate message-oriented middleware for abstracting and encapsulating functionality. To ensure the architectural capabilities, the experiment devised highlighted the hypothesis that message latency between two independent services is lower if the communication is inferred between one broker instead of two and it should at maximum have one second of lead time. The experiment was a simulated toy model to validate performance while the primary metric to be accounted for, was the timeliness of data exchange for end-to-end latency between services. The evidence from the experiment confirmed that message latency between both a MQTT and Kafka broker is higher in contrast to only using one. The increase of timeliness were insubstantial, which underlines that the current architectural design deployed is both optimal and complies adequately with the stated requirements and quality attributes.