\section{Problem, research questions, and approach}

The objective of this project is to design a complex system within the Industry 4.0 domain, and work with different architectures and technologies to implement a prototype. The system design must satisfy the following requirements: \linebreak

\begin{enumerate}
    \item Production software must be able to exchange and coordinate information to execute a production and change production.
    \item Production software must run 24/7.
    \item Production software must be continuously deployable.
\end{enumerate}

Furthermore, the software must address the interoperability, deployability, and availability quality attributes.
The focus of the project will be to design a Lego recycling production system, that can collect and sort recycled Lego bricks and package them for customers. This includes designing and implementing a software architecture prototype, that can support the stated requirements for the production system. Additionally, it should support the interaction between production components which enables collecting, sorting, and, finally, packaging the Lego bricks. This leads to the following problem definition:
\textit{"How can a production system be designed where the production components interact with each other to collect, sort, and package recycled Lego bricks?".}
The stated problem definition leads to the following research questions:
\begin{enumerate}
    \item How can different architectures support the stated production system requirements?
    \item Which architectural trade-offs must be taken due to the technology choices?
\end{enumerate}

The stated research questions will be answered by the architectural research conducted, whereas decisions will be taken for designing and implementing the architecture accordingly.
The initial approach for the system architecture will be to define and propose assumptions for the production system and define use cases. Then, the architecture of the Industry 4.0 domain will be explored by identifying the system and subsystems for the production and how they should interact with each other. This includes identifying and selecting suitable programming languages, databases, and message bus technologies that support the architecture to be used for implementing the prototype.








