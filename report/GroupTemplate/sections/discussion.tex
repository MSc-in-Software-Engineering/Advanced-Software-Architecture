\section{Discussion and Future work}

The current approach includes implementation of the microservice architecture which can be extended to include dynamic scaling capabilities based on real-time demand. We can implement different auto-scaling mechanisms that can automatically adjust the number of instances of microservices to handle fluctuations in workload, which will further ensure optimal resource utilisation and responsiveness during varying production demands. 


Additionally, the current evaluation focuses on the performance of the system which measures the message latency between two services that are inferred between brokers. Future work could involve testing the scalability of the architecture across diverse production settings, considering variations in the size and complexity of the production system. This would provide insights into the system’s performance under different operational conditions. Similarly, the evaluation of interoperability of the system can be extended to include stress testing scenarios where the system operates under high loads. It is assessed how well the system maintains interoperability under stress conditions, ensuring that the system can handle workloads without compromising communication and collaboration between components.


