\section{Literature review}
To accommodate the research questions to be answered and a review of the state of the art for software architecture, a thorough systematic literature review was conducted. The literature review consolidated of a search process, inclusion/exclusion criteria, data extraction, and quality assessment, which proposed synthesis to clarify results. It is essential to know that the approach partaken was not to initialize a study. However, it was to both review and analyse upon the sources in relevance to contextualizing new knowledge to the field of state of the art within software architecture. The search process focused primarily on finding academic sources relevant to the technological development through an abstract and citation database. Additionally, the inclusion/exclusion criteria proposed required crucial guidelines such as only taken literature written in English into account, age of literature and to mainly filter by abstract as well as conclusion. As the data were to be extracted and quality assessed, the sources were examined through rigorousness and relevance analysis, by assessing their research methods and the impact of the research respectively. The approach accumulated suitable sources to be reviewed through narrative synthesis, by combining the primary findings of studies in a textual format through summaries.
\subsection{Designing and Evaluating Interoperable Industry 4.0 Middleware Software Architecture: Reconfiguration of Robotic System}
The study by Worm, T; and Jepsen, C, S; focused on designing as well as evaluating upon Industry 4.0 middleware software architecture, in regard to a quality attribute scenario defined as interoperability. The evaluation conducted by the authors was on actual production equipment, which proposed that the design of a proper middleware software architecture fulfilled an initialized interoperability scenario. It is concluded that proposing middleware software architecture is one qualified solution to accommodate production rearrangement in a suitable manner\cite{Jepsen2023205}.
\subsection{Evaluation of the Maintainability Aspect of Industry 4.0 Service-oriented Production}
The study by Schnicke, F and Esper, K states quick changeable production is a crucial objective of Industry 4.0, which in perspective, focuses on adapting manufacturing to be more alterable and efficient through the interconnections of different production components. The results indicate that Industry 4.0 proposes less change impact to implement changed scenarios than in Industry 3.0. This is mainly due to the capabilities that Industry 3.0 requires additional communicative links to transfer data between components, that in the end can affect the developed architecture. Industry 4.0 relies solely upon the concept of digital twins to share data between production components\cite{Esper20208}.
\subsection{Industry 4.0 Middleware Software Architecture Interoperability Analysis}
The study by Hviid, J; Mørk, I, T; Worm, T; and Jepsen, C, S refers that middleware interoperability have a crucial part within the networking of processes and production machines for Industry 4.0. The study analyses upon the implications of several levels of interoperability in middleware software architecture. It concludes that the levels of interoperability in middleware software architecture proposes concerns that are eligible to be divided into components, that can handle the levels respectively and efficiently. The authors further states it will enable proper clarification in regard to the impact that can stem from environmental characteristics\cite{Jepsen202132}. 
\subsection{Exploring Architectural Design Decisions in Industry 4.0: A Literature Review and Taxonomy}
The paper by Voss, S; Dorofeev, K; and Terzimehic focused on investigating both valid and optimal architectural design decisions in respect to Industry 4.0 clarifications. The primary findings were that deployment and production planning respectively were the most performed design decisions. The timing elements were the main consideration in Industry 4.0 challenges, while aspects within the qualities of reliability and reconfigurability were virtually abandoned. Domain-specific figures were furthermore the most intriguing utility for modelling architecture\cite{Terzimehić202287}. 
\subsection{A Reconfigurable Industry 4.0 Middleware Software Architecture}
The study by Worm, T; Siewertsen, B; and Jepsen, C, S; aims to both design and evaluate upon a reconfigurable Industry 4.0 middleware software architecture with a developed quality attribute scenario in relation to reconfigurability. It is stated that based on concluded results of the architecture, that it is yet another solution to support flexible production, as it constitutes of relevant architectural capabilities. Additionally, the study proposes promising demonstrations of specifying reconfigurable quality attribute scenarios as well as the design of architecture, that supports flexible Industry 4.0 production systems\cite{Jepsen202343}.
\subsection{Understanding and addressing quality attributes of microservices architecture}
The paper by Babar, A, M; Shen, J; Shan, Z; Zhang, C; Zhong, C; Jia, Z; Zhang, H; and Li, S aimed at investigating the empirical state-of-the-art of quality attributes for systems with microservice architecture. The authors state that the microservice architecture prevails the limitations of architectures of a monolithic nature. From the identified tactics, scalability is the most common quality of such architecture; however, other quality attributes should be explored for trade-offs i.e., within performance. Furthermore, the qualities of maintainability is an aspect that requires attention for improvement in the future\cite{Li2021}.
\subsection{Trends in continuous evaluation of software architectures}
The study by Nakagawa, Y, E; Santos, V; Capilla, R; and Soares, C,  R aspires that the software profession faces the urgency for systems that can handle continuous deployment, whereas the concept of continuous software engineering is introduced. The authors state that even though systems rely on intuitive software architectures, architectural evaluation is equally important for the quality assurance of such architectures. Architectural evaluation should be integrated as part of the general software development cycles, which have not been adapted yet to par with Industry 4.0\cite{Soares20231957}.
\subsection{Software architectures of the convergence of cloud computing and the Internet of Things}
The paper by Oivo, M; Kuvaja, P; Pakanen, O; and Banijamali, A, focuses on identifying state-of-the-art architectural capabilities as well as evaluation in relation to the Internet of Things and cloud computing. It was identified that the service-oriented architecture was the most common pattern in regard to the relations mentioned, whereas the scalability, timeliness, and security were the most identified quality attributes. The conclusion is that research within the software architecture domain for the Internet of Things and cloud computing is increasing\cite{Banijamali2020}. 
\subsection{Addressing the review}
To summarise the results, there are four different architectures that could support flexible production systems in relation to Industry 4.0, which are 1) \textit{middleware software architecture}, 2)  \textit{microservice architecture},  3) \textit{reconfigurable middleware software architecture} and to some extent 4) \textit{service-oriented architecture}. Each architecture proposes architectural tradeoffs, and should be reviewed extensively e.g., through continuous software engineering, whereas key quality attributes have to be considered and evaluated upon.